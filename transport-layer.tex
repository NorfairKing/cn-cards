\documentclass[main.tex]{subfiles}
\begin{document}
\small

\layer{Transport}

\begin{card}[Transport Control Protocol]{TCP}
\item Features
    \begin{itemize}
        \item Streams
        \item Connections
        \item Arbitrary length data
        \item Flow control and Congestion control
    \end{itemize}
\item Three-way handshake: $Syn(x)$, $Syn(y)Ack(x+1)$, $Ack(y+1$
\item Symmetric connection release
\end{card}

\begin{card}[User Datagram Protocol]{UDP}
\item Features
    \begin{itemize}
        \item Messages
        \item Connectionless
        \item Stateless
    \end{itemize}
\end{card}

\begin{card}{Socket}
\item Full name: Berkley Sockets
\item Abstraction of transport layer API
\item Sockets let apps attach to a port
\end{card}

\begin{card}{Port}
\item The end-point for transport layer traffic: TSAP
\end{card}

\begin{card}{Segment}
\item The unit of data in the transport layer
\end{card}

\begin{card}[Maximum Segment Size]{MSS}
\item The largest valid size for a TCP segment.
\end{card}

\begin{card}[ISN]{Initial Sequence Number}
\item First valid sequence number to use in TCP
\item Agreed on (by both sides) in three way handshake.
\end{card}

\begin{card}[Transport Service Access Point]{TSAP}
\item Defines an end-point for transport layer traffic
\item Example: Ports
\end{card}

\begin{card}{Stop-and-Wait}
\item Send one segment and wait for response, then repeat
\item Disadvantage: very inefficient when bandwith-delay is much larger than one packet.
\end{card}

\begin{card}{Sliding Window}
\item Send $2\times bandwith \times delay$ packets and wait for responses, then repeat.
\item Advantage: uses network optimally.
\item Variations
    \begin{itemize}
        \item Go back $N$: Receiver only acks last valid packet and discard rest.
        \item Selective repeat: Receiver buffers out-of-order packets and acks all that fit in buffer.
    \end{itemize}
\end{card}


\end{document}

%%% Local Variables:
%%% mode: latex
%%% TeX-master: t
%%% End:
