\documentclass[main.tex]{subfiles}
\begin{document}
\small

\layer{Transport}

\begin{card}[Transport Control Protocol]{TCP}
\item Features
    \begin{itemize}
        \item Reliable bidirectional bytestream
        \item Connections
        \item Arbitrary length data
        \item Flow control and Congestion control
    \end{itemize}
\item Three-way handshake: $Syn(x)$, $Syn(y)Ack(x+1)$, $Ack(y+1$
\item Symmetric connection release
\item Only unicast
\end{card}

\begin{card}{Delayed Acknowledgements}
\item Used in TCP
\item Delay sending an acknowledgement to send a higher cumulative acknowledgement
\item Interacts badly with Nagle's algorithm
\item Works best on very reliable connection.
\end{card}

\begin{card}{Nagle's Algorithm}
\item Send multiple very small messages at once.
\item Buffer as long as last packet hasn't been acknowledged and packet is not full.
\item Interacts badly with delayed Acknowledgements.
\end{card}

\begin{card}[User Datagram Protocol]{UDP}
\item Features
    \begin{itemize}
        \item Messages
        \item Connectionless
        \item Stateless
    \end{itemize}
\item Advantages: relatively low overhead
\end{card}

\begin{card}{Socket}
\item Full name: Berkley Sockets
\item Abstraction of transport layer API
\item Sockets let apps attach to a port
\end{card}

\begin{card}{Port}
\item The end-point for transport layer traffic: TSAP
\item Used to uniquely name a transport layer client.
\end{card}

\begin{card}{Segment}
\item The unit of data in the transport layer
\end{card}

\begin{card}[Maximum Segment Size]{MSS}
\item The largest valid size for a TCP segment.
\end{card}

\begin{card}[ISN]{Initial Sequence Number}
\item First valid sequence number to use in TCP
\item Agreed on (by both sides) in three way handshake.
\end{card}

\begin{card}[Transport Service Access Point]{TSAP}
\item Defines an end-point for transport layer traffic
\item Example: Ports
\end{card}

\begin{card}{Stop-and-Wait}
\item Send one segment and wait for response, then repeat
\item Disadvantage: very inefficient when bandwith-delay is much larger than one packet.
\end{card}

\begin{card}{Sliding Window}
\item Send $2\times bandwith \times delay$ packets and wait for responses, then repeat.
\item Advantage: uses network optimally.
\item Variations
    \begin{itemize}
        \item Stop and wait: one frame at a time.
        \item Go back $N$: Receiver only acks last valid packet and discard rest.
        \item Selective repeat: Cumulatieve acks and nacks.
    \end{itemize}
\end{card}

\begin{card}{Adaptive Timeout}
\item Makes sure that most of the packets' acks arrive within timeout.
\item Keep smoothed estimates of RTT and variance in RTT.
\item Set new timeout to $estimate + 4\times Variance$
\end{card}

\begin{card}{Min-Max Fairness}
\item Definition: an allocation of bandwidth is min-fair if the bandwidth given to one flow cannot be increased without decreasing bandwidth for a smaller flow.
\end{card}

\begin{card}[Additive Increase, Multiplicative Decrease]{AIMD}
\item Increase flow additively while network is not congested, decrease multiplicatively when congestion occurs.
\item Ensures quickest convergence to optimal flow.
\item Used in TCP
\end{card}

\begin{card}[Real-Time Protocol]{RTP}
\item designed for end-to-end, real-time, transfer of streaming media
\end{card}

\begin{card}[Real-Time Control Protocol]{RTCP}
\item Used to provide QoS feedback to RTP
\end{card}

\begin{card}{Sliding Window Ack Clocking}
\item A part of a sliding window
\item Automatically spread packets according to the speed of the bottleneck link.
\item Helps network run with low levels of loss and delay.
\end{card}

\begin{card}{TCP Slow Start}
\item Before AIMD
\item Double the congestion window after every round-trip time untill congestion occurs (notice via ack timeout).
\item Concretely: Increase congestion window by one packet for every ack received.
\item We want to get the congestion window near to the right rate quickly, but additive increase is too slow and a fixed congestion window is too arbitrary.
\item Slow start is only slow when compared to the conversion of a fixed congestion window. It's really very fast (exponential).
\end{card}

\begin{card}{Fast retransmit and Fast recover}
\item The MD part of AIMD in TCP
\item Fast retransmit: Treat three duplicate acks as a loss
\item Fast recover: After fast retransmit, start MD
\end{card}

\begin{card}[Selective Acknowledgement]{SACK}
\item Includes, not only the cumulative acknowledgement of segments up until a certain point, but a notice of which segments have been received.
\end{card}

\begin{card}[Explicit Congestion Notification]{ECN}
\item An extension of TCP/IP to allow for notification of network congestion
\item Allows for communication without dropping packets
\end{card}



\end{document}

%%% Local Variables:
%%% mode: latex
%%% TeX-master: t
%%% End:
