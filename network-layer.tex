\documentclass[main.tex]{subfiles}
\begin{document}
\small

\layer{Network}

\begin{card}{Packet}
\item The unit of data in a network layer.
\end{card}

\begin{card}{Routing}
\item The process of deciding in which direction to send traffic.
\item Network wide
\end{card}

\begin{card}{Forwarding}
\item The process of sending a packet on its way.
\end{card}

\begin{card}[Internet Protocol]{IP}
\TODO
\end{card}

\begin{card}{Datagram Model}
\item Each datagram contains a destination address.
\item No setup needed.
\item Router state: per destination
\item Routing: per packet
\item Failures are easy to mask
\item Hard to add quality of service
\end{card}

\begin{card}{Virtual Circuit Model}
\item Don't contain destination address, just labels.
\item Setup needed
\item Router state: per connection
\item Routing: per circuit
\item Failures are not easy to mask
\item Easy to add quality of service
\end{card}

\begin{card}[Multiple Protocol Label Switching]{MPLS}
\TODO
\end{card}

\begin{card}[Time To Live]{TTL}
\item Decremented with every router hop.
\item Throws ICMP error if ever hits zero.
\end{card}

\begin{card}{Longest matching prefix}
\TODO
\end{card}

\begin{card}{Hierarchical Routing}
\item For routing efficiency: reduced routing table size
\item Disadvantage: Sometimes packets take longer paths.
\end{card}

\begin{card}[Maximum Transmission Unit]{MTU}
\item The largest packet which will fit through this network.
\end{card}

\begin{card}{IP packet fragmentation}
\item Necessary because the entire packet might not fit into a frame.
\item Drawbacks:
    \begin{itemize}
        \item Increases the processing load on routers that have to fragment a packet.
        \item Decreases the probability of an entire packet being received.
    \end{itemize}
\item Alternatives: MTU discovery
\end{card}

\begin{card}[Internet Control Message Protocol]{ICMP}
\item Reports packet forwarding errors to sender.
\item On top of IP
\item Used in MTU discovery (destination unreachable due to fragmentation)
\end{card}

\begin{card}{IPV4}
\item 32 bit addresses
\end{card}

\begin{card}{IPV6}
\item 128 bit addresses
\end{card}

\begin{card}{Middlebox}
\item Advantages: Control over many hosts and possible rapid deployment.
\item Disadvantages: breaking layering interferes with connectivity and poor vantage point for tasks.
\end{card}

\begin{card}[Network Address Translation]{NAT}
\item Is a middlebox.
\item Connects an internal network to an external network.
\item Motivated by ipv4 address scarcity.
\item Makes ISP think that hosts are just different (TCP) ports on one IP address by rewriting IP headers.
\item Advantages: partially solves the IP address scarcity problem
\item Disadvantages: doesn't work on connectionless protocols (UDP), breaks apps that expose IP adresses (FTP)
\end{card}

\begin{card}{Routing}
\item Good routing is comes down to finding a good bandwith allocation.
\item A \textbf{distributed} and \textbf{decentralized} way of `finding the right path to take' for packets.
\item Goals:
    \begin{itemize}
        \item Correctness
        \item Efficiency
        \item Fairness
        \item Fast convergence
        \item Scalability
    \end{itemize}
\end{card}

\begin{card}{Unicast}
\item A delivery model
\item Deliver traffic to one target.
\end{card}

\begin{card}{Broadcast}
\item A delivery model
\item Deliver traffic to all targets
\end{card}

\begin{card}{Multicast}
\item A delivery model
\item Deliver traffic to a group of targets
\end{card}

\begin{card}{Anycast}
\item A delivery model
\item Deliver traffic to one target, from the best source.
\end{card}


\begin{card}{Shortest path routing}
\item Route packets according to shortest path
\item Assign every link a cost (should represent ...)
\item Has the optimality property.
\end{card}

\begin{card}{Optimality property}
\item If a path is optimal, then its subpaths are optimal as well.
\item Shortest paths have this property.
\end{card}

\begin{card}{Distance Vector Routing}
\item Distributed shortest paths
\item Each node maintains a vector of distances and sends it to neighbors to update.
\item Disadvantage: slow convergence
\end{card}

\begin{card}{Count to infinity scenario}
\item In distance vector routing
\item Happens when node disappears
\end{card}

\begin{card}[Routing Information Protocol]{RIP}
\item Uses distance vector routing
\item Includes split horizon, poison reverse heuristic
\item Infinity is 16 hops $\Rightarrow$ no `count to infinity' scenarios
\end{card}

\begin{card}{Flooding}
\item Broadcast a message through an entire network
\item Every node: send the incoming message to all other neighbors. + remember the message
\item Disadvantage: Inefficient because one node can receive message several times.
\end{card}

\begin{card}[Link State Packet]{LSP}
\item Contains information on a network topology.
\end{card}

\begin{card}{Link State Routing}
\item Trades more computation for better dynamics.
\item 2 phases:
    \begin{enumerate}
        \item Nodes flood local topology in link state packets (LSP) so that each node learns the full topology.
        \item Each node computes its own forwarding table with a shortest path algorithm. (Dijkstra)
    \end{enumerate}
\item Notes next to failed node/link will flood updated LSPs.
\end{card}

\begin{card}[Open Shortest Path First]{OSPR}
\item Modern link state protocol.
\end{card}

\begin{card}[Intermediate System to Intermediate System]{IS-IS}
\item Modern link state protocol.
\end{card}

\begin{card}[Multi Path Routing]{MPR}
\item Just to increase performance by using multiple paths
\end{card}

\begin{card}[Equal Cost Multi Path routing]{ECMP}
\item MPR but only use paths with equal cost.
\item Uses source DAG instead of source tree.
\item Advantage: easier to balance load.
\item Disadvantage: introduces ordering issues.
\end{card}

\begin{card}[Border Gateway Protocol]{BGP}
\item Computes interdomain routes
\item Path vector protocol
\item Border routers announce routes to eachother
\end{card}

\begin{card}[Autonomous System]{AS}
\item Party in the BGP setting (example: ISP)
\end{card}

\begin{card}[Network Service Access Point]{NSAP}
\item Defines an end point for network layer traffic
\item Example: IP Addresses
\end{card}

\end{document}

%%% Local Variables:
%%% mode: latex
%%% TeX-master: t
%%% End:
