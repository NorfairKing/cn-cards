\documentclass[main.tex]{subfiles}
\begin{document}
\small

\layer{Data-Link}

\begin{card}{Frame}
\item Unit of data in the link layer.
\item Structure: header,data,trailer
\item Destination address first in header so that receivers can see whether the packet is for them.
\end{card}

\begin{card}{Byte count framing}
\item Method for framing
\item Start the beginning of each frame with a length field.
\item Problem: no way of resynchronising after an error.
\end{card}

\begin{card}{Byte stuffing}
\item Method for framing
\item Add a flag to the beginning and end of each frame and escape any data that contains that flag.
\end{card}

\begin{card}{Bit stuffing}
\item Flag of six consecutive flags
\item Sender and receivers fix the message to escape flags
\item Good: Less overhead than with byte stuffing.
\item Problem: Frame lengths are not fun to work with.
\end{card}

\full{Point-to-Point Protocol}
\begin{card}{PPP}
\item Used to frame IP packets over bytestreams.
\item Uses byte stuffing.
\end{card}

\full{Cyclic Redundancy Check}
\begin{card}{CRC}
\item Error detection with cyclic codes.
\item Used in Ethernet, 802.11, ADSL, Cable,...
\end{card}

\full{(Assymmetric) Digital Subscriber Line}
\begin{card}{(A)DSL}
\item Used to transmit data over telephone lines.
\item It uses the otherwise unused higher frequency bands for data.
\end{card}

\full{Automatic Repeat reQuest}
\begin{card}{ARQ}
\item Used when errors are common or need to be corrected: Wifi, TCP, ...
\item Receiver acknowledges correct frames.
\item Sender resends after timeout.
\end{card}

\full{}
\begin{card}{Stop and Wait}
\item ARQ with one bit sequence numbers.
\end{card}

\begin{card}{Multiplexing}
\item Just a fancy word for sharing, usually a network.
\item Good for continuously used channels.
\end{card}

\full{Time Division Multiplexing}
\begin{card}{TDM}
\item Sharing the network (channel/link) over time.
\end{card}

\full{Frequency Division Multiplexing}
\begin{card}{FDM}
\item Sharing the network by putting users on different frequency bands.
\item Used in television and radio stations.
\end{card}

\full{Randomized Multiple Access Control}
\begin{card}{RMAC}
\item A strategy for controlling which node gets to used a shared link.
\item Basic idea: Send when you have traffic, resend randomly later if collision.
\end{card}

\full{}
\begin{card}{ALOHA}
\item Computer network for Hawaian islands.
\item Send when you have traffic, resend randomly later if collision.
\item Very simple, works well under low load.
\item Highly inefficient under high load.
\item Slotted version increases maximum efficiency from 18\% to 36\%
\end{card}

\full{Carrier Sense Multiple Access}
\begin{card}{CSMA}
\item Like ALOHA, but listens before sending.
\item This is easy when wired, not so much wireless.
\item Still possible to have collisions because of delay.
\item Only a good defense against collisions when bandwidth-delay product is less than one packet.
\item Very simple, works well under low load.
\item Highly inefficient under high load.
\end{card}

\full{Carrier Sense Multiple Access with Collision Detection}
\begin{card}{CSMA/CD}
\item Reduces the cost of collisions by detecting them and aborting (jam) rest of frame time.
\item Time window for node to hear of collision is $2\times delay$
\item Impose a minimum frame size that takes at least $2\times delay$ seconds to send so a node can't finish before a collision if one is going to happen.
\end{card}

\begin{card}{Classic ethernet}
\item All nodes share one coaxial cable.
\item Uses 1-persistent CSMA/CD with BEB
\item No Ack or retransmissions
\item IEEE 802.3
\end{card}

\full{Binary Exponential Backoff}
\begin{card}{BEB}
\item Heuristic for estimating how long to wait after a frame collision by estimating the amount of nodes on a link.
\item BEB doubles the waiting interval for each successive collision.
\item Very efficient in practice because the wait time quickly gets large enough to work.
\end{card}

\full{}
\begin{card}{Modern ethernet}
\item Based on switches
\item Also called switched ethernet.
\end{card}

\full{Wireless Multiple Access Control}
\begin{card}{WMAC}
\item Nodes can't hear when sending, can't detect collisions: No CSMD/CD!
\end{card}

\full{Multiple Access Control with Collision Avoidance}
\begin{card}{MACA}
\item Short handshake:
    \begin{enumerate}
        \item Sender: Request to send? (RTS) (with frame length)
        \item Receiver: Clear to send! (CTS) (with frame length)
        \item Sender transmits while other nodes that heard CTS stay quiet.
    \end{enumerate}
\item Solves both exposed terminals and hidden terminals.
\end{card}

\full{Access Point}
\begin{card}{AP}
\item Access point for connectivity with rest of internet in a WiFi schema.
\end{card}

\full{}
\begin{card}{WiFi}
\item Has a MAC problem for access to access point
\item CSMA/CA, RTS/CTS optional.
\item Acks and retransmissions.
\item Three addresses due to AP
\item IEEE 802.11
\end{card}

\full{Carrier Sense Multiple Access Controll with Collision Avoidance}
\begin{card}{CSMA/CA}
\item Like MACA, but avoids collisions by inserting small random gaps instead of RTS/CTS.
\end{card}

\full{Contention-Free Multiple Access}
\begin{card}{CFMA}
\item Based on 'turn taking': Token is passed around.
\item Fixed (higher) overhead with no collisions: predictable service.
\item Better than CSMA under high load.
\item More complex than CSMA, more things can go wrong.
\end{card}

\full{}
\begin{card}{Switch}
\item Basis of modern (switched) ethernet.
\item Uses frame addresses to connect input to output ports.
\item Multiple frames may be switched in parallel.
\item Need buffers so loss is possible.
\item More reliable than classic ethernet.
\item Offers scalable performance.
\end{card}

\full{}
\begin{card}{Backward learning}
\item The way that switches find out where to send frames.
\item To fill the table: look at source adresses on input frames.
\item To forward: look up or else broadcast.
\item Works with any number of switches and hubs if there are no loops in the topology (or it uses a switch spanning tree).
\end{card}

\full{}
\begin{card}{Switch Spanning Tree}
\item Switches collectively agree on spanning tree.
\item Allows for loops in topology.
\end{card}

\full{Network Allocation Vector}
\begin{card}{NAV}
\item Used in WiFi 802.11
\item Each frame carries a NAV field field that says how long the transmission sequence will take. (Frame + Ack)
\end{card}

\end{document}

%%% Local Variables:
%%% mode: latex
%%% TeX-master: t
%%% End:
