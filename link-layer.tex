\documentclass[main.tex]{subfiles}
\begin{document}
\small

\layer{Link}

\begin{card}{Frame}
\item Unit of data in the link layer.
\item Structure: header,data,trailer
\end{card}

\begin{card}{Byte count framing}
\item Method for framing
\item Start the beginning of each frame with a length field.
\item Problem: no way of resynchronising after an error.
\end{card}

\begin{card}{Byte stuffing}
\item Method for framing
\item Add a flag to the beginning and end of each frame and escape any data that contains that flag.
\end{card}

\begin{card}{Bit stuffing}
\item Flag of six consecutive flags
\item Sender and receivers fix the message to escape flags
\item Good: Less overhead than with byte stuffing.
\item Problem: Frame lengths are not fun to work with.
\end{card}

\full{Point-to-Point Protocol}
\begin{card}{PPP}
\item Used to frame IP packets over bytestreams.
\item Uses byte stuffing.
\end{card}

\full{Cyclic Redundancy Check}
\begin{card}{CRC}
\item Error detection with cyclic codes.
\item Used in Ethernet, 802.11, ADSL, Cable,...
\end{card}

\full{(Assymmetric) Digital Subscriber Line}
\begin{card}{(A)DSL}
\item Used to transmit data over telephone lines.
\item It uses the otherwise unused higher frequency bands for data.
\end{card}

\full{Automatic Repeat reQuest}
\begin{card}{ARQ}
\item Used when errors are common or need to be corrected: Wifi, TCP, ...
\item Receiver acknowledges correct frames.
\item Sender resends after timeout.
\end{card}

\full{}
\begin{card}{Stop and Wait}
\item ARQ with one bit sequence numbers.
\end{card}

\begin{card}{Multiplexing}
\item Just a fancy word for sharing, usually a network.
\item Good for continuously used channels.
\end{card}

\full{Time Division Multiplexing}
\begin{card}{TDM}
\item Sharing the network (channel/link) over time.
\end{card}

\full{Frequency Division Multiplexing}
\begin{card}{FDM}
\item Sharing the network by putting users on different frequency bands.
\item Used in television and radio stations.
\end{card}

%\begin{card}{}
%\end{card}

\end{document}

%%% Local Variables:
%%% mode: latex
%%% TeX-master: t
%%% End:
